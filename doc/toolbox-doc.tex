\documentclass[a4paper,fleqn]{article}
%% Seitenaufbau
\usepackage[top=3cm, bottom=2.5cm, left=3.5cm, right=3.5cm]{geometry}

%% Schriftbild
\usepackage{lmodern}  % Latin Modern Zeichensatz
\usepackage[utf8]{inputenc}  % Unterstuetzung von Umlauten im Quelltext
\usepackage[T1]{fontenc}  % Korrekte Umlaute im Output
\usepackage[ngerman]{babel}  % Silbentrennung nach neuer Rechtschreibung
\renewcommand{\familydefault}{\sfdefault}  % Serifenlose Schrift
\usepackage{setspace}\onehalfspacing  % 1.5-facher Zeilenabstand
\renewcommand{\arraystretch}{1.5}  % 1.5-facher Zeilenabstand (Tabellen)
\setlength{\parindent}{0pt}  % Keine Einrueckung am Beginng von Absaetzen
\sloppy  % Weniger strikte Silbentrennung

%% Verlinkung von Inhaltsverzeichnis, Bildern und Formeln
\usepackage[pagebackref]{hyperref}  % Verlinkung von URLs und Referenzen
\usepackage{color}  % Definition von Linkfarben
\definecolor{DarkRed}{rgb}{0.5,0,0}
\hypersetup{
  colorlinks,
  citecolor=DarkRed,
  linkcolor=DarkRed,
  urlcolor=blue}

%% Inhalt der Titelseite
\title{Einführung in die Python Toolbox}
\author{}
\date{\today}

%%%%%%%%%%%%%%%%%%%%%%%%%%%%%%%%%%%%%%%%%%%%%%%%%%%%%%%%%%%%%%%%%%%%%%
\begin{document}
\maketitle

\section*{Konzept}
Der Werkzeugkasten bietet einen Einstieg in die Verarbeitung und Darstellung von
Datensätzen mit Python 3.5. Das Ziel ist nicht die Bereitstellung fertiger
Lösungen für spezielle Problemstellungen. Viel mehr werden die Nutzer an
grundlegende Arbeitsschritte in Python herangeführt.

Zu diesem Zweck steht eine Vielzahl von Pythonskripten zur Verfügung, die in
kurzen Anwendungsbeispielen wichtige Funktionen und Auswertungskonzepte
vorstellen. Hierbei steht die Anschaulichkeit im Vordergrund: Jedes Skript ist
ausführbar und unterstützt durch sichtbare Ergebnisse das Verständnis des
Quellcodes.

\subsection*{Anwendungsbeispiel}
Die angedachte Nutzung des Werkzeugkastens zur Lösung einer Aufgabenstellung
soll nun an einem kurzen Beispiel erläutert werden.

Gegeben sei folgende Aufgabenstellung:

\begin{quote}
''Die Daten der Temperaturanpassung eines Thermometers an seine Umgebung liegen
in einer ASCII-Datei vor. Lesen Sie die Daten ein und bestimmen sie die
Zeitkonstante des Thermometers, indem Sie eine Exponentialfunktion an den
Verlauf der Messdaten anpassen. Plotten Sie sowohl die Messsdaten als auch die
gefittete Exponentialfunktion.''
\end{quote}

Der erste Schritt bei der Lösung einer solchen Problemstellung ist es, das
Problem in kleiner Teilaufgaben zu unterteilen. In unserem Beispiel lassen sich
folgende Teilaufgaben definieren:

\begin{enumerate}
  \item Das Einlesen der Messdaten.
  \item Das Anpassen einer Exponentialfunktion an einen Datensatz.
  \item Das Plotten der Messdaten und der Exponentialfunktion.
  \item Das Speichern des Plots als Bilddatei.
\end{enumerate}

Sollten bei der Bearbeitung dieser Teilschritte Probleme auftreten, kann in der
Toolbox gezielt nach Hilfestellungen für die Teilaufgaben gesucht werden. Somit
wird verhindert, dass der Nutzer viel Zeit bei der Suche auf Antworten nach
unspezifischen Fragen verliert .

\newpage

Für die Lösung des Anwedungsnbeispiels sind folgende Abschnitte der Toolbox
relevant:

\begin{itemize}
  \item Einlesen einer ASCII Datei Grundlagen.
  \item Exponential fit.
  \item Plotten mehrerer Linien.
  \item Speichern von Abbildungen.
\end{itemize}

Die jeweiligen Beispielskripte können nun als Bausteine genutzt werden, um ein
eigenes Pythonskript zu schreiben, in welchem die Teilaufgaben abgearbeitet
werden.\\

Bei Problemen, die auch unter Verwendung der Toolbox nicht zu lösen sind, gilt:

\begin{center}
  \textbf{DON'T PANIC!}
\end{center}

Ein Blick ins Internet\footnote{In 99\% der Fälle ist hiermit
\url{http://www.stackoverflow.com} gemeint.} oder die Dokumentationen der
genutzten Python Pakete kann in vielen Fällen helfen.  Man muss das Rad nicht
immer neu erfinden.

% TODO: Einen schönen Abschlussatz finden.
\end{document}
